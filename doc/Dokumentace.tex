\documentclass[a4paper, 11pt]{article}

\usepackage[czech]{babel}
\usepackage[utf8]{inputenc}
\usepackage[left=2cm, top=3cm, text={17cm, 24cm}]{geometry}
\usepackage{times}
\usepackage[unicode]{hyperref}
\usepackage{indentfirst}
\usepackage{graphics}
\usepackage{fancyvrb}
\usepackage{listings}
\usepackage{xcolor}
\lstset{basicstyle=\footnotesize\ttfamily,breaklines=true}
\lstset{framextopmargin=50pt,frame=none}
\hypersetup{colorlinks = true, hypertexnames = false}

\renewcommand*\contentsname{Obsah}

\begin{document}

	\begin{titlepage}
		\begin{center}
			\LARGE\textsc{Vysoké učení technické v~Brně} \\
			\Large\textsc{Fakulta informačních technologií}\\
			\vspace{\stretch{0.382}}
			\large{IFJ - Projektová dokumentace} \\
			\LARGE{Implementace překladače jazyka IFJ22} \\
			\vspace{0.2cm}
			\large{Tým "Tým xkalis03", varianta BVS}
			\vspace{\stretch{0.618}}
		\end{center}

		\Large{\hspace*{-0.6cm}Autoři: \hfill (v) \hspace{0.1cm} Vojtěch Kališ (xkalis03)} \hspace{0.97cm} 25\% \\
		\Large{\hspace*{\fill} Jan Lutonský (xluton02)} \hspace{0.82cm} 25\% \\
		\Large{\hspace*{\fill} Jan Salaš (xsalas02)} \hspace{1.73cm} 25\% \\
		\Large{\hspace*{\fill} Lucie Hlaváčová (xhlava60)} \hspace{0.1cm} 25\% \\ \\
		\Large{Implementovaná rozšíření:  \hspace*{1cm} FUNEXP} \hfill
	\end{titlepage}

%%% TOC
	\tableofcontents

%%%
	\newpage
	\section{Úvod}
	Cílem tohoto projektu bylo vytvořit překladač implementovaný v~jazyce C, který ze standartního vstupu načte vstupní kód napsaný v~jazyce IFJ22, přeloží
	jej do cílového jazyka IFJcode22 a výskedek pak vypíše na standartní výstup. Jazyk IFJ22 vznikl jako obdoba jazyka PHP. 
%%%
	\section{Implementace}
	Celý překladač jsme si rozdělili na více dílčích problémů, jejichž funkčnost byla individuálně testována. Tyhle dílčí problémy byly nadále vzájemně 
	propojovány a opět testována jejich funkčnost.
	\subsection{Lexikální analyzátor}
	\subsection{Syntaktický analyzátor}
	\subsubsection{Precedenční syntaktický analyzátor}
	\subsection{Sémantický analyzátor}
	\label{semantic}
	Sémantický analyzátor provádí sémantickou analýzu nad vstupním programem, a je obsažen v souboru \textit{semantic\.c}; jeho hlavičkový soubor pak 
	analogicky nese název \textit{semantic\.h}. Sémantický analyzátor pracuje převážně s Globální tabulkou symbolů (využívající implementace 
	\hyperref[symtab]{Tabulky symbolů} a Abstraktním syntaktickým stromem (dále jen ASS). Očekává se korektní naplnění ASS v rámci syntaktické analýzy. Na 
	začátku své funkce sémantický analyzátor inicializuje Globální tabulku symbolů, projde ASS a vyhledá v něm všechny definice funkcí, jež vloží jakožto nody s 
	typem \textit{function} do Globální tabulky symbolů; možné parametry definované při deklaraci funkce zase vloží do Lokální tabulky symbolů dané funkce 
	jakožto nody s typem \textit{variable}. Do Globální tabulky symbolů jsou také vloženy deklarace vestavěných funkcí (zavoláním funkce ), společně s funkcí 
	nazvanou ":b" sloužící jako hlavní tělo programu (\textit{body}).

	Jakmile je vše připraveno, Sémantický analyzátor vstoupí do funkce \textit{AST\_DF\_traversal}, plnící funkci hlavní smyčky, která prochází již zmíněný AST 
	do hloubky a v rámci switch case-u pak hledá AST nody, jejichž sémantickou korektnost je třeba prověřit; jakmile nějakou takovou nodu najde, spustí nad ní 
	speciální funkci zabývající se prověřením sémantické korektnosti toho konkrétního typu AST nody. V případě, že je nalezena sémantická chyba, program 
	ukončí svou činnost, vypropaguje kód odpovídající nalezené chybě, a zaručí, že dojde ke kompletnímu uvolnění veškeré alokované paměti.
	\subsection{Generátor cílového kódu}
%%%
	\section{Speciální datové struktury}

	\subsection{Tabulka symbolů}
	\label{symtab}
	Tabulka symbolů byla implementována jako binární vyhledávací strom, což bylo i nárokem naší varianty zadání. Téměř celá tabulka symbolů je napsána 
	nerekurzivním (tedy iterativním) postupem, a to především z důvodu snížení časové komplexity na úkor složitější implementace. Tabulka symbolů je 
	využívána \hyperref[semantic]{Sémantickým analyzátorem} pro vytvoření Globální tabulky symbolů a její využití je již popsáno v rámci jeho popisu.
	\subsection{Obousměrně vázaný seznam}
	Implementaci obousměrně vázaného seznamu lze najít v souboru \textit{dll.c}, a odpovídající hlavičkový soubor pak pod názvem \textit{dll.h}. Obousměrně 
	vázaný seznam je v projektu využíván ve struktuře nody Tabulky symbolů, a to pro účely snadného uchování názvů argumentů vkládaných funkcí.
	\subsection{ADT \#3}
	\subsection{...}
%%%
	\section{Práce v týmu}
	Na projektu jsme začali pracovat ihned po zveřejnění zadání, a to jeho prostudováním a domluvením první schůzky, v rámci které byla vypracována prvotní 
	verze pravidel LL-gramatiky, a LL tabulka. Obojí se ještě v čase dalších několika týdnů upravovalo v případě nalezení chyby, až se nakonec vše ustálilo do 
	konečné podoby, prezentované v tomhle dokumentě a dohledatelné na jeho \hyperref[gram]{konci}.
	
	Dále jsme si jednotlivé části rozdělili mezi sebe a pracovali na nich jako jednotlivci popřípadě dvojice. Stále probíhaly schůzky, například pro řešení 
	implementačních záležitostí a to především způsobu komunikace jednotlivých částí překladače, ovšem zvětšiny docházelo spíše k průběžnému testování 
	překladače jako celku a kontrole pokroku ve vývoji.
	\subsection{Komunikace}
	Pro komunikaci byla využita aplikace \textit{Discord}, kde byl vytvořen vlastní server na kterém pak probíhala veškerá vzájemná komunikace, ať už se 
	jednalo o komunikaci textovou, hovorové schůzky celotýmové i třeba v menším počtu, nebo sdílení materiálů, diagramů apod. 
	\subsection{Vzdálený repozitář}
	Jako vzdálený repozitář jsme zvolili Github, jenž nám umožnil sdílet mezi sebou zdrojové kódy dílčích úkolů a~navzájem si testovat nejen funkčnost 
	jednotlivých částí, ale i~překladač jako celek.
	\subsection{Rozdělení práce v týmu}
	Rozdělení: \hspace*{0.5cm} 	Jan Lutonský (xluton02): \hspace{0.1cm} Syntaktická analýza, Precedenční syntaktická analýza, Generátor \\
			\hspace*{5cm} cílového kódu, Speciální datové struktury, Testování\\
			\hspace*{2.25cm}	Vojtěch Kališ (xkalis03): \hspace{0.1cm} Sémantická analýza, Tabulka symbolů, Speciální datové struktury, \\
			\hspace*{5cm} Testování\\
			\hspace*{2.25cm}	Jan Salaš (xsalas02): \hspace{0.1cm} Lexikální analýza, Testování\\
			\hspace*{2.25cm}	Lucie Hlaváčová (xhlava60): \hspace{0.1cm} Lexikální analýza, Testování\\
%%%
	\section{Závěr}
	
%%%
	\newpage
	
	\begingroup\centering
	\section*{DKA pro konečný automat}
	\endgroup
	\vspace{1.8cm}
	\begin{figure}[h]
		\centering
		\hspace*{-1.2cm}
		\scalebox{0.5}{\includegraphics{diagrams/DKA.png}}
		\label{DKA}
	\end{figure}
%%%
	\newpage

	\begingroup\centering
	\section*{Precedenční tabulka}
	\endgroup
	\vspace{1.8cm}
	\begin{figure}[h]
		\centering
		\hspace*{-2cm}
		\scalebox{0.7}{\includegraphics{diagrams/precedence_table.png}}
		\label{prec_table}
	\end{figure}
%%%
	\newpage
	
	\begingroup\centering
	\section*{Pravidla LL gramatiky}
	\label{gram}
	\endgroup
	\begin{lstlisting}[language=Python]
	prog -> PS_MARK PC_MARK prog_body

	prog_body -> body_part prog_body
	           | fun_def prog_body
	           | EPS
	
	prog_end -> PE_MARK EOS
	          | EOS
	
	body -> body_part body
	      | EPS
	
	body_part -> if_n
	           | while_n
	           | extended_expr
	           | ret
	
	extended_expr -> EXPR SEMIC 
	               | EXPR_FCALL SEMIC
	               | EXPR_PAR SEMIC
	               | EXPR_ASSIGN SEMIC
	
	ret -> RETURN ret_cont
	ret_cont -> EXPR SEMIC
	          | EXPR_PAR SEMIC
	          | EXPR_FCALL SEMIC
	          | SEMIC
	
	while_n -> WHILE EXPR_PAR LBRC body RBRC
	
	if_n -> IF EXPR_PAR LBRC body RBRC else_n
	else_n -> ELSE LBRC body RBRC
	        | EPS
		  
	fun_def -> FUNC F_ID LPAR par_list RPAR TYPE ret_type LBRC fun_body RBRC
	
	par_list -> type_n ID par_list_cont
	          | EPS
	
	par_list_cont -> COMMA par_list
	               | EPS
				
	ret_type -> type_n
	          | VTYPE
	
	type_n -> STYPE
	        | ITYPE
		| FTYPE
		| NSTYPE
		| NITYPE
		| NFTYPE

	\end{lstlisting}

%%%

\end{document}
