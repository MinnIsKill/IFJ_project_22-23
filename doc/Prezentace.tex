%%%%%%%%%%%%%%%%%%%%%%%%%%%%%%%%%%%%%%%%%
% Beamer Presentation
% LaTeX Template
% Version 1.0 (10/11/12)
%
% This template has been downloaded from:
% http://www.LaTeXTemplates.com
%
% License:
% CC BY-NC-SA 3.0 (http://creativecommons.org/licenses/by-nc-sa/3.0/)
%
% Modified by Jeremie Gillet in March 2017 to make an OIST template
%
%%%%%%%%%%%%%%%%%%%%%%%%%%%%%%%%%%%%%%%%%

% !!!!!
%	I HEREBY SWEAR NOT TO USE THIS TEMPLATE FOR ANY MEANS WHICH WOULD 
%     IN ANY WAY LEAD TO A MONETARY GAIN
% !!!!!

% Author (aside from template, obviously): Vojtech Kalis (xkalis03), presentation created 
%                                                                    for a project for subject IFJ, VUTBR FIT

%----------------------------------------------------------------------------------------
%	PACKAGES AND THEMES
%----------------------------------------------------------------------------------------

\documentclass[10pt]{beamer}

\usepackage{graphicx} % Allows to include images
\usepackage{booktabs} % Allows the use of \toprule, \midrule and \bottomrule in tables
\usepackage{textcomp}

\mode<presentation> {

\usetheme{default}

\usecolortheme[named=white]{structure} % White titles and such
\setbeamercolor{normal text}{fg=white} % White text
\setbeamercolor{background canvas}{bg=black} % Black background
\setbeamertemplate{itemize item}{\color{white}$\bullet$} % Comment this line for default bullet points (triangles)
\usepackage{helvet}
\renewcommand{\familydefault}{\sfdefault}

\setbeamertemplate{navigation symbols}{} % No navigation symbols
\setbeamertemplate{footline}
 {\begin{minipage}{125mm} \vspace{-4 mm} \hfill \textbf{\normalsize{\insertframenumber\,/\,\inserttotalframenumber}} \end{minipage}}
 }

%----------------------------------------------------------------------------------------
%	TITLE PAGE
%----------------------------------------------------------------------------------------

\title[Short title]{Prezentace k projektu do předmětu IFJ\\\vspace{0.3cm}}
\subtitle{Tým "Tým xkalis03", varianta BVS}
\author{Vojtěch Kališ, Jan Lutonský, Jan Salaš, Lucie Hlaváčová}
\date{}

\begin{document}

\setbeamertemplate{background}{\includegraphics[width=\paperwidth, trim = 0 0 0 -17]{img/title.png}} % Adding the background logo for the title page

\begin{frame}[plain]
\vspace*{1.55cm}
\titlepage
\end{frame}

\setbeamertemplate{background}{\includegraphics[width=\paperwidth]{img/background.png}} % Adding the background logo for the rest of the slides

\begin{frame}
\frametitle{Obsah}
\centering
\textbf{IMPLEMENTACE} \\
\pause
\vspace{0.5cm}
\textbf{SPECIÁLNÍ DATOVÉ STRUKTURY} \\
\pause
\vspace{0.5cm}
\textbf{PRÁCE V TÝMU} \\
\pause
\vspace{0.5cm}
\textbf{ZÁVĚR}

\end{frame}

%----------------------------------------------------------------------------------------
%	PRESENTATION SLIDES
%----------------------------------------------------------------------------------------

\begin{frame}
\frametitle{Implementace}
\begin{itemize}[<+->]
\item struktura Context
	\begin{itemize}[<2->]
	\item $\rightarrow$ context.c, context.h
	\end{itemize}
\pause
\vspace{0.2cm}
\item Lexikální analýza
	\begin{itemize}[<4->]
	\item $\rightarrow$ lex.c, lex.h
	\item 3 módy, stavy jako funkce
	\item funkce lex\_init, lex\_next
	\end{itemize}
\pause
\vspace{0.2cm}
\item Syntaktická analýza
	\begin{itemize}[<6->]
	\item $\rightarrow$ parser.c, parser.h, parser\_expr.c, parser\_expr.h
	\item LL-gramatika, rekurzivní sestup
	\item precedenční syntaktická analýza
	\item AST
	\end{itemize}
\end{itemize}
\end{frame}

\begin{frame}
\frametitle{Implementace}
\begin{itemize}[<+->]
\item Sémantická analýza
	\begin{itemize}[<2->]
	\item $\rightarrow$ semantic.c, semantic.h
	\item globální tabulka symbolů, definice funkcí
	\item práce s AST
	\end{itemize}
\pause
\vspace{0.2cm}
\item Generování cílového kódu
	\begin{itemize}[<4->]
	\item $\rightarrow$ codegen.c, codegen.h
	\item architektura, průchod AST
	\item implicitní typové konverze
	\end{itemize}
\end{itemize}
\end{frame}

%------------------------------------------------

\begin{frame}
\frametitle{Speciální datové struktury}
\begin{itemize}[<+->]
\item Abstraktní syntaktický strom
	\begin{itemize}[<2->]
	\item $\rightarrow$ ast.c, ast.h
	\item N-ární strom
	\item Použití
	\end{itemize}
\pause
\vspace{0.2cm}
\item Generický zásobník
	\begin{itemize}[<4->]
	\item $\rightarrow$ g\_stack.c, g\_stack.h
	\item dynamické pole ukazatelů na void
	\item přialokování místa
	\end{itemize}
\pause
\vspace{0.2cm}
\item Zásobník pro precedenční syntaktickou analýzu
	\begin{itemize}[<6->]
	\item $\rightarrow$ g\_stack.c, g\_stack.h
	\item přetypovaný generický zásobník
	\item pro ukládání uzlů syntaktického stromu
	\end{itemize}
\pause
\vspace{0.2cm}
\item Tabulka symbolů
	\begin{itemize}[<8->]
	\item $\rightarrow$ symtable.c, symtable.h
	\item binární strom
	\item nerekurzivní implementace
	\end{itemize}
\end{itemize}
\end{frame}

%------------------------------------------------

\begin{frame}
\frametitle{Práce v týmu}
\begin{itemize}[<+->]
\item Komunikace
	\begin{itemize}[<2->]
	\item Discord
	\item GitHub
	\end{itemize}
\pause
\vspace{0.2cm}
\item Rozdělení práce
	\begin{itemize}[<4->]
	\item teorie společně
	\item následné rozdělení na podúkoly
	\end{itemize}
\end{itemize}
\end{frame}

%------------------------------------------------

\begin{frame}
	\vspace{\stretch{0.382}}
	\Large{\centerline{PROSTOR PRO DOTAZY}}
	\vspace{\stretch{0.618}}
\end{frame}



%------------------------------------------------

\end{document} 